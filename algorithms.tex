\documentclass[a4paper,10pt,twocolumn]{extarticle}
\usepackage[a4paper , top=8mm, right=8mm,left=8mm, bottom=8mm]{geometry}
\usepackage{nopageno}
\thispagestyle{empty}
\usepackage{geometry}
\usepackage{amsmath} % for math and equations
\usepackage{amssymb} % for symbols
\usepackage[boxruled,noend]{algorithm2e}
\usepackage{algorithmicx} %algorithm
\usepackage{algpseudocode}
\usepackage{hyperref}
%declarations

\algnewcommand{\algorithmicand}{\textbf{ and }}
\algnewcommand{\algorithmicor}{\textbf{ or }}
\algnewcommand{\OR}{\algorithmicor}
\algnewcommand{\AND}{\algorithmicand}
\algnewcommand{\var}{\texttt}
\usepackage{comment}
\usepackage{color}
\usepackage{float}
\usepackage{hyperref}
\SetAlCapNameFnt{\scriptsize}
\SetAlCapFnt{\scriptsize}
\renewcommand{\abstractname}{\vspace{-\baselineskip}}
\setlength{\belowcaptionskip}{-10pt}

%Algorithm2e Declarations
\SetKwProg{Func}{function}{:}{{end}}
\SetKwInput{In}{in}
\SetKwInput{Not}{not}
\SetKwInput{Return}{return}
\SetKwInput{Declare}{declare}
\SetKwInput{Set}{set}
\SetKwInput{KwInput}{Input} % Set the Input
\SetKwInput{KwOutput}{Output} % set the Output
\SetKwInput{KwPre}{pre} % Set the Pre
\SetKwInput{KwPost}{post} % set the Post

\definecolor{pagecolor}{rgb}{1,0.98,0.9}
\title{\vspace{-10mm} \footnotesize \textbf{REFERENCE ALGORITHMS} \vspace{-14mm}}
\date{\scriptsize \today \hspace{0.5mm} (v0.1) \vspace{-4mm}}

\algnewcommand\Signal{\textbf{signal}}
\begin{document}
\maketitle
\scriptsize
\begin{abstract}
\noindent \textbf{Disclaimer :} The below psedocodes provide basic instructions for the implementation of bitcoin blink protocols. It is constantly reviewed and updated by core-programmers and co-authors. Revisit for new versions.
\end{abstract}
\begin{algorithm}[h]
\scriptsize
\DontPrintSemicolon
\caption{Network Graph}
\SetKwFunction{AddPeerToGraph}{addPeerToGraph}
\SetKwFunction{RemovePeerFromGraph}{removePeerFromGraph}
\SetKwFunction{SignAndSendRandomMessage}{signAndSendRandomMessage}
\SetKwFunction{ReceiveAndVerifyRandomMessage}{receiveAndVerifyRandomMessage}
\SetKwFunction{UpdateNetworkGraph}{updateNetworkGraph}
\textbf{var} networkGraph = \{\}\; 
\tcc{an empty object to represent the network graph}
\textbf{var} transactions = [ ]\;
\Func{\AddPeerToGraph(peer)}{
networkGraph[peer.publicKey] = peer\;
}
\Func{\RemovePeerFromGraph(peer)}{
delete networkGraph[peer.publicKey]\;
}
\Func{\SignAndSendRandomMessage(peer)}{
\textbf{var} randomMessage = generateRandomMessage()\;
\textbf{var} signedMessage = signMessage(randomMessage)\;
sendMessage(peer, signedMessage)\;
}
\Func{\ReceiveAndVerifyRandomMessage(peer, message)}{
\textbf{var} verified = verifyMessage(peer.publicKey, message)\;
\If{(verified)}{
updateNetworkGraph(peer, message)\;
}}
\Func{\UpdateNetworkGraph(peer, message)}{
 networkGraph[peer.publicKey] = \{   
    publicKey: peer.publicKey,\;
    address: peer.address,\;
    status: `online',\;
    lastMessage: message,\;
    lastUpdated: Date.now() \}\;
}
\end{algorithm}

\begin{algorithm}
\scriptsize
\DontPrintSemicolon
\caption{Network Graph Transaction functions}
\SetKwFunction{SendTransaction}{sendTransaction}
\SetKwFunction{Addproducer}{add\_producer}
\SetKwFunction{ProcessTransactions}{processTransactions}
\SetKwFunction{ConstructTransaction}{constructTransaction}
\Func{\SendTransaction(origin, destination, message)}{
\tcc{Construct the transaction with path and encrypted instructions}
\textbf{var} transaction = constructTransaction(origin, destination, message)\;
\tcc{Add the transaction to the list of unconfirmed transactions}
transactions.push(transaction)\;
}
\tcc{Function to process unconfirmed transactions and add them to blocks}
\Func{\ProcessTransactions()}{
\textbf{var} producers = getProducers()\;
\For{(var i = 0; i $<$ transactions.length; i++)}{
\textbf{var} transaction = transactions[i]\;
\tcc{Add the transaction to each producer's block}
\For{(var j = 0; j $<$ producers.length; j++)}{
\textbf{var} producer = producers[j]\;
\textbf{var} added = addTransactionToBlock(producer, transaction)\;
\If{(added)}{
\tcc{Remove the transaction from the list of unconfirmed transactions}
transactions.splice(i, 1)\;
i-\;
\textbf{break}\;
}
}
}
}
\Func{\ConstructTransaction(origin, destination, message)}{
\tcc{Find the shortest path between the origin and destination}
\textbf{var} shortestPath = findShortestPath(origin, destination)\;
\textbf{var} encryptedMessage = encryptMessage(message, destination.publicKey)\;
 \textbf{var} transaction = \{
 
    path: shortestPath,\;
    instructions: encryptedMessage,\;
  \}
\tcc{Return the constructed transaction}
\textbf{return} transaction\;
}
\tcc{Function to add a new producer to the network}
\Func{\Addproducer(public\_key, allocated\_blocks)}{
producer = (public\_key, allocated\_blocks)\;
producers.append(producer)\;
}

\end{algorithm}

\begin{algorithm}
\scriptsize
\DontPrintSemicolon
\caption{Network Graph Topology}
\SetKwFunction{Gettopology}{get\_topology}
\tcc{Function to get the network topology from a given reference node}
\Func{\Gettopology(reference\_node)}{
topology = [ ]\;
\For{node \textbf{in} nodes}{
\If{node != reference\_node}{
path = shortest\_path(reference\_node, node)\;
topology.append((node, path))\;
}
}
\textbf{return} topology\;
}
\end{algorithm}

\begin{algorithm}[h]
\scriptsize
\DontPrintSemicolon
\caption{Path Finding}
\SetKwFunction{Findpath}{findPath}
\Func{\Findpath(fromNode, toNode)}{
\tcc{Find a route from Node to Node}
paths = getAllPaths(fromNode)\;
routes = [ ]\;
\For{path \textbf{in} paths}{
\tcc{Check if the path is connected to the destination node}
\If {path.toNode == toNode}{
\textbf{return} [path]\;
\tcc{Try to find a route from the destination node through this channel}
route = findPath(path.toNode, toNode)\;
\If{route is \textbf{not} None}{
\tcc{Add this path to the route}
routes.append([path] + route)\;}}
\tcc{Return the route}
\If{len(routes) $>$ 0}{\textbf{return} (routes)\;\Else{\textbf{return} None\;}}}}
\end{algorithm}

\begin{algorithm}[h]
\scriptsize
\DontPrintSemicolon
\caption{Onion Peeling}
\SetKwFunction{Onionpath}{onion\_path}
\Func{\Onionpath(mint\_hash, route)}{
\tcc{Get the next hop path in the route}
next\_path = route.pop()\;
packet = create\_onion\_packet(mint\_hash, next\_path)\;
\For{path \textbf{in} reversed(route)}{
eph\_key = generate\_ephemeral\_key()\;
packet = add\_path\_to\_onion\_route(path, eph\_key, packet)\;
}
send\_packet\_to\_next\_hop\_path(packet, next\_path)\;
response = receive\_response\_from\_next\_hop\_path()\;
\For{path \textbf{in} reversed(route)}{
response = decrypt\_response\_with\_ephemeral\_key(response, path, eph\_key)\;
}
\textbf{return} response\;
}
\vspace{2mm}
\tcc{Notes : The onion peeling algorithm is used to protect the privacy of the mint route, by encrypting the mint information multiple times, with each layer containing information for the next hop. As the payment packet is passed from hop to hop, each node removes a layer of encryption to reveal the next hop in the route.
\begin{itemize}
\item mint\_hash is the unique identifier for the minted transaction
\vspace{-0.2cm}
\item route is a list of the nodes in the mint route
\vspace{-0.2cm}
\item add\_path\_to\_onion\_packet function adds a new layer to the onion packet for the current hop
\vspace{-0.2cm}
\item ephemeral key will be used to decrypt the response from that hop \end{itemize}}
\end{algorithm}

\begin{algorithm}[h]
\scriptsize
\DontPrintSemicolon
\caption{Node Weights}
bandwidth = x\;
block\_size\_limit = 1000000 \tcc{in bytes}\;
node\_weights = \{\}\;
\tcc{Scan the blockchain from the genesis block to the current block}
\For {\textbf{each block} in blockchain}{
proof\_utxo = get\_bandwidth\_proof\_utxo(\textbf{block})\;
proof\_data = get\_proof\_data(proof\_utxo)\;
node\_weight = calculate\_weight(proof\_data, bandwidth)\;
fork\_proof = get\_fork\_proof(\textbf{block})\;
\If{fork\_proof \textbf{is not} None}{
prover\_node = fork\_proof.prover\_node\;
forker\_node = fork\_proof.forker\_node\;
node\_weights[prover\_node] += 0.01 $*$ block\_size\_limit\;
node\_weights[forker\_node] -= 0.01 $*$ block\_size\_limit\;
}
\tcc{Add the node weight to the temporary storage for the current node}
node\_weights[\textbf{block}.node\_id] = node\_weight\;
\textbf{continue}\;
}
\end{algorithm}

\begin{algorithm}[h]
\scriptsize
\DontPrintSemicolon
\SetKwFunction{Addnewblock}{add\_new\_block()}
chain = [ ]\;
ring\_size = 1\;
block\_size\_limit\_per\_sec = 0\;
set\_weights = [ ]\;
confirm\_snips = false\;
\Func{\Addnewblock}{
\textbf{new}\_\texttt{block} = get\_new\_\texttt{block}()\;
last\_block = get\_last\_block(chain)\;
\textbf{new}\_\texttt{hash\_proof} = last\_block.hash\_proof\;
\textbf{new}\_\texttt{block}.hash\_proof = new\_\texttt{hash\_proof}\;
\If{\textbf{new}\_\texttt{hash\_proof}.node\_weight $>$ last\_block.hash\_proof.node\_weight}{
\tcc{Find the snips to remove by linearly hashing one by one snip}
\textbf{new}\_\texttt{snips} = last\_block.snips\;
\For{snip \textbf{in} last\_block.snips}{
\If{linear\_hash(snip) == \textbf{new}\_\texttt{hash\_proof}.MCR\_output}{
\textbf{break}\;
}
\textbf{new}\_\texttt{snips}.remove(snip)\;
}
\textbf{new}\_\texttt{block}.snips =  \textbf{new}\_\texttt{snips}\;
}
\If {block\_time(\textbf{new}\_\texttt{block}) \OR block\_size\_capped(\textbf{new}\_\texttt{block}) \OR end\_snip(\textbf{new}\_\texttt{block})}{
chain.append(\textbf{new}\_\texttt{block})\;
}
}
\caption{Adding new block}
\end{algorithm}

\begin{algorithm}[h]
\scriptsize
\DontPrintSemicolon
\SetKwFunction{Setringsizenewblock}{set\_ring\_size(new\_block)}
\SetKwFunction{Setringval}{set\_ring\_validators()}
\caption{Set new ring Validators}
\Func{\Setringsizenewblock}{
\If{is\_confirmed(\textbf{new}\_\texttt{block})}{
\If{is\_forked(\textbf{new}\_\texttt{block})}{
ring\_size -= 1\;
end\_election()\;
}
\Else{
ring\_size += 1\;
tail\_join\_req = 2\;
set\_ring\_size(ring\_size)\;
}
return ring\_size\;
}
}
\Func{\Setringval}{
set\_weights = sorted(nodes, key=lambda node: \texttt{node}[``\texttt{weight}"], reverse=True)\;
set\_weights = [n \textbf{for} n \textbf{in} set\_weights \textbf{if} n not \textbf{in} prev\_ring\_validators]\;
prev\_ring\_validator\_weights = [n.weight \textbf{for} n \textbf{in} prev\_ring\_validators \textbf{if} n.weight $\geq$ 0]\;
mean\_weight = mean(prev\_ring\_validator\_weights)\;
tail\_join = mean\_weight\;
k = calculate\_MD160hash(\textbf{new}\_\texttt{block})\;
set\_weights = [n \textbf{for} n \textbf{in} set\_weights \textbf{if} n.bandwidth $>$ tail\_join]\;
\tcc{Current hex should be lesser than k}
Valid\_keys = [ ]\;
\For{i in range(len(set\_weights))}{
\If{set\_weights[i].pubkey.hex $<$ k}{
valid\_keys.append(set\_weights[i].pubkey)\;
}
}
Rand1, rand2 = get2\_random\_numbers\_in\_range(0, len(valid\_keys)-1)\;
pubkeys.append(valid\_keys[rand1])\;
pubkeys.append(valid\_keys[rand1])\;
\tcc{If none , take immediate greater 2 values}
\If{pubkeys is None}{
Valid\_keys = sorted(nodes, key=lambda node: \texttt{set\_weights}, reverse=false)\;
pubkeys.append(valid\_keys[0]) \;
pubkeys.append(valid\_keys[1])\;
}
ring\_validators = set\_weights\;
ring\_validators.append(keys \textbf{for} key \textbf{in} pubkeys)\;
}
\end{algorithm}

\begin{algorithm}[h]
\scriptsize
\DontPrintSemicolon
\SetKwFunction{ConfirmSnips}{confirm\_snips()}
\caption{Confirm Snips}
\Func{\ConfirmSnips}{
block\_size\_limit = min([node.bandwidth \textbf{for} node \textbf{in} ring\_validators])\;
block\_times = [block.time \textbf{for} block \textbf{in} previous\_blocks]\;
block\_time\_median = median(block\_times)\;
per\_block\_size = block\_size\_limit $*$ block\_time\_median\;
per\_block\_time\_count = int(per\_block\_size / ihr)\;
\tcc{Set a cap that not more than the individual block time the producer should produce}
max\_individual\_block\_time\_count = cap\_value\;
\If{per\_block\_time\_count $>$ max\_individual\_block\_time\_coun}{
per\_block\_time\_count = max\_individual\_block\_time\_count - 1\;
confirm\_snips = true\;
}
\Else{confirm\_snips = false\;}
}
\end{algorithm}

\begin{algorithm}
\scriptsize
\DontPrintSemicolon
class MerkleChain\;
\KwPre{the snip is added to the data}
\KwPost{the data is added to the chain}
 add\_node(snip)\;
 d $\gets$ snip\;
  \uIf{head = null}{
    head,tail $\gets$ add\_data(d)\;}
    \Else {tail $\gets$ add\_data(d)\;}
\hrulefill
class add\_data(d)\;
\KwPre{the value is added to the vector}
\KwPost{the vector is generated to a merkle tree and added to the chain}
New Vector data\;
data $\gets$ d\;
\If{size(data) == max\_block\_size}{generate\_root(data)}
\hrulefill
generate\_root()\;
\KwPre{the vector data is added as the leaves}
\KwPost{merkel tree and its root is generated }
New Vector temp\_data\;
temp\_data $\gets$ data\;
\While {temp\_data $>$ 1}{
\For{i = 0  i $<$ size(temp\_data)  i+2}{
Left $\gets$ temp\_data[i]\;
Right $\gets$ (i+1 == size(temp\_data)) ? temp\_data[i] : temp\_data[i+1]\;
combined = Left + Right \;
new\_temp\_data $\gets$ hash(combined)\;
}
temp\_data $\gets$ new\_temp\_data\;
}
node\_root $\gets$ temp\_data[0]\;
\hrulefill
main()\;
\SetKwInput{KwInitialized}{initialized} % set the Initialized
\KwInitialized{chain is an object of class MerkleChain and  string data}
\While {true}
{Output “enter data (q to quit)”
Get data\;
\If {data = q}
{ 
\textbf{break}\;
\Else {
addnode(data)\;}
      }
     }
\caption{Merkle Chain}
\end{algorithm}


\begin{algorithm}[h]
\scriptsize
\DontPrintSemicolon
\caption{Hash Proofs : helper functions}
\SetKwFunction{RejectSnips}{reject\_snips()}
\SetKwFunction{AcceptSnips}{accept\_snips()}
\Func{\RejectSnips}{
new\_block\_hash = produce\_block(prev\_block\_hash, current\_block\_snips, current\_block\_time)\;
send\_block\_to\_network(new\_block\_hash)\;
\tcc{Reset variables for new block}
current\_block\_snips = []\;
current\_block\_size = 0\;
current\_block\_time = 0\;
prev\_block\_hash = new\_block\_hash\;
snips\_received = false\;
}
\Func{\AcceptSnips}{
\tcc{single threaded hash concatenate}
routing\_instruction = get\_routing\_instruction()\;
snip\_data = receive\_snip\_data()\;
preimage = generate\_preimage(snip\_data, prev\_snip\_hash)\;
signature = sign\_preimage(preimage)\;
hashed\_data = hash(concatenate(preimage, signature))\;
send\_snip\_to\_next\_node(routing\_instruction, hashed\_data)\;
current\_block\_snips.append(hashed\_data)\;
current\_block\_size += get\_snip\_size(hashed\_data)\;
current\_block\_time = get\_current\_block\_time()\;
prev\_snip\_hash = hashed\_data\;
mcr = produce\_mcr(snips)\;
block\_header.add(mcr)\;
snips\_received = true\;
}
\end{algorithm}

\begin{algorithm}[h]
\scriptsize
\DontPrintSemicolon
\caption{Hash Proofs}
prev\_snip\_hash = null\;
prev\_block\_hash = genesis\_block\_hash\;
current\_block\_size = 0\;
current\_block\_snips = [ ]\;
current\_block\_time = 0\;
block\_size\_limit\_per\_sec = initial\_block\_size\_limit\_per\_sec\;
snips\_received = confirm\_snips() \tcc{snips\_algo-3}
\While{true}{
\If{snips\_received}{
\If{current\_block\_size $\geq$ block\_size\_limit\_per\_sec $*$ current\_block\_time \OR current\_block\_time $\geq$ individual\_block\_time\_cap}{
reject\_snips()\;
}
\tcc{Move on to next snip}
current\_snip = next\_snip()\;
\Else{accept\_snips()}
}
\Else{accept\_snips()}
}

\end{algorithm}

\begin{algorithm}[h]
\scriptsize
\DontPrintSemicolon
\caption{Hash Reward}
initial\_reward = 50 $*$ 10$**$8 \tcc{example 50 BTC}
halving\_period = 210\_000 \tcc{example blocks}
\tcc{Set the starting block height and the total number of remaining blocks}
block\_height $=$ 0\;
remaining\_blocks = halving\_period\;
percent\_hash\_rate $=$ 0\;
all\_nodes\_IHR = 100000 \tcc{example total IHR of all nodes}
\While{true}{
\tcc{Calculate the total number of remaining coins and remaining hashes}
remaining\_coins = initial\_reward $*$ remaining\_blocks\;
remaining\_hashes = remaining\_blocks $*$ 1.26 $*$ 10$**$8\;
percent\_hash\_rate = get\_node\_IHR()/all\_nodes\_IHR\;
\tcc{Calculate the reward per block and the reward per hash}
reward\_per\_block = remaining\_coins / remaining\_blocks\;
\If{fork\_slash}{
reward\_per\_hash = (remaining\_coins / remaining\_hashes) $*$ (percent\_hash\_rate)\;
}
\Else{remaining\_coins = remaining\_coins + (remaining\_coins / remaining\_hashes) $*$ (percent\_hash\_rate)}
\tcc{Check if it's time to halve the rewards}
\If {remaining\_blocks $\leq$ 0}{
\textbf{break}\;}
\tcc{Halve the remaining blocks and update the block height}
remaining\_blocks \tcc{= 2}
block\_height += halving\_period\;
}
\end{algorithm}

\begin{algorithm}[h]
\scriptsize
\DontPrintSemicolon
\caption{Transfer Fee}
transfer\_fee = 0.0005 \tcc{Transfer fee should be \textbf{in range} 0.0005 \textbf{to} 0.00005}

\Func{check\_range(transaction\_fee)}{
	\If{(transaction\_fee $>$ 0.0005)}{
               \textbf{return}  0.0005\;}
      \If{(transaction\_fee $<$ 0.00005)}{
               \textbf{return} 0.00005\;}
}

\tcc{called \textbf{for} every nth \textbf{block} , where n \textbf{is} the ring size}
\Func{transaction\_fee()}{

     \tcc{Find total volume of all blocks of all tokens with their exchange rate}
       ring\_volume1 = get\_volume\_of\_tokens(\textbf{block} $\rightarrow$ previous)\;
       ring\_volume2 = get\_volume\_of\_tokens(\textbf{block} $\rightarrow$  previous $\rightarrow$ previous)\;

       \If{standard\_deviation(ring\_volume1, ring\_volume2) $\geq$ 0.75}{
          \If{ring\_volume1 $>$ ring\_volume2}{
		transfer\_fee += 0.000005\;
            transfer\_fee = check\_range(transaction\_fee)\;
}
          \If{ring\_volume1 $<$  ring\_volume2}{
		transfer\_fee -= 0.000005\;	
            transfer\_fee = check\_range(transaction\_fee)\;
}
}
       \Else{continue}
}

\end{algorithm}


\begin{algorithm}[h]
\scriptsize
\DontPrintSemicolon
\SetKwInput{KwSigIn}{signal input} % set the Signal Input
\SetKwInput{KwSigOut}{signal output} % set the Signal Input
\tcc{Public signals}
\KwSigIn{node\_ihr}
\KwSigIn{ihr\_hash}
\tcc{Private signals}
\KwSigIn{salt}
\KwSigIn{required\_ihr}
\tcc{Output signal}
\KwSigOut{if\_pass}
\tcc{Range proof check}
\textbf{signal}  buffer\;
\textbf{signal}  range\_check\;
\If{node\_ihr $>$ required\_ihr - buffer \AND node\_ihr $<$ required\_ihr + buffer} {range\_check = true\;}
\tcc{Verify hash}
\textbf{signal}  hash\;
\textbf{signal}  hash\_check\;
\tcc{RIPEMD160 to calculate the hash}
hash = RIPEMD160 (salt, required\_ihr)\;
\If{hash == ihr\_hash} {hash\_check = true\;}
\uIf{range\_check \AND hash\_check} {if\_pass = true\;}
\Else{if\_pass = false\;}
\tcc{Bandwidth circuit $\equiv$ IHR circuit}
\caption{ZK IHR Circuit}
\end{algorithm}

\begin{algorithm}[h]
\scriptsize
\DontPrintSemicolon
\caption{Open Order Script Deploy}
\SetKwFunction{Order}{order}
\SetKwFunction{Claim}{claim}
\textbf{declare} token\_a \textbf{as} integer\;
\textbf{declare} seller \textbf{as} PubKey\;
\textbf{declare} token\_b \textbf{as} integer\;
\textbf{declare} mature\_time \textbf{as} integer\;
\textbf{set} mature\_time \textbf{as} expiry\_time\;
\Func{\Order(sig, b, buyer, current\_exchange\_rate\_value, preimage)}{
\If{mature\_time $>$ SigHash.nLocktime(preimage)}{
\If{checkSig(sig, buyer)}{
\If{Tx.checkPreimage(preimage)}{
\If{b == this.token\_b}{
scriptCode = SigHash.scriptCode(preimage)\;
codeend = 104\;
codepart = scriptCode[:104]\;
outputScript\_send = codepart + buyer + num2bin(this.token\_a, 8) + num2bin(current\_exchange\_rate\_value, 8) + num2bin(tds, 8)\;
output\_send = Utils.writeVarint(outputScript\_send)\;
outputScript\_receive = codepart + this.seller + num2bin(this.token\_b, 8) + num2bin(current\_exchange\_rate\_value, 8) + num2bin(tds, 8)\;
output\_receive = Utils.writeVarint(outputScript\_send)\;
hashoutput = hash256(output\_send+output\_receive)\;
\If{hashoutput == SigHash.hashOutputs(preimage)}{
\tcc{order is open \& placed}
}
}
}
}
}
}
\end{algorithm}

\begin{algorithm}[h]
\scriptsize
\DontPrintSemicolon
\caption{Open Order Claim}
\SetKwFunction{Claim}{claim}
\Func{\Claim(sig, value, pubKey, current\_exchange\_rate\_value, preimage)}{
\If{mature\_time $<$ SigHash.nLocktime(preimage)}{
\If{pubKey == this.seller}{
\If{checkSig(sig, pubKey)}{
\If{Tx.checkPreimage(preimage)}{
\If{value == this.token\_a}{
scriptCode = SigHash.scriptCode(preimage)\;
codeend = 104\;
codepart = scriptCode[:104]\;
outputScript\_claim = codepart + pubKey + num2bin(this.token\_a,8) + num2bin(current\_exchange\_rate\_value,8) + num2bin(tds, 8)\;
output\_claim = Utils.writeVarint(outputScript\_claim)\;
hashoutput = hash256(output\_claim)\;
\If{hashoutput == SigHash.hashOutputs(preimage)}{
\tcc{claim is successful}
}
}
}
}
}
}
}
\end{algorithm}

\begin{algorithm}[h]
\scriptsize
\DontPrintSemicolon
\caption{Bitcoin Exchange \& Demand Rate}
\SetKwFunction{Updatetokenpricelist}{update\_token\_price\_list}
\SetKwFunction{CalBDR}{cal\_bdr}
\Func{\Updatetokenpricelist(open\_order\_list: List[List[str]]) $\gets$ Dict[str, Dict[str, float]]}{
token\_price\_dict = \{\}\;
\For{\textbf{each} order \textbf{in} open\_order\_list}{
token\_pair = order[0]\;
token\_id = token\_pair.split (`/')[0]\;
bitcoin\_rate = float (order[1])\;
token\_rate = calculate\_mid\_market\_price (float(order[2]), float(order[3]))\;
percentage\_movement = calculate\_percentage\_movement (float(order[4]), token\_rate)\;
\If{token\_pair \textbf{not} \textbf{in} token\_price\_dict}{
token\_price\_dict[token\_pair] = \{`exchange\_rate': token\_rate, `percentage\_movement': percentage\_movement\}\;
}
\Else{
token\_price\_dict[token\_pair][`exchange\_rate'] = token\_rate\;
token\_price\_dict[token\_pair][`percentage\_movement'] = percentage\_movement\;
}
}
\textbf{return} token\_price\_dict\;
}
\Func{\CalBDR(token\_price\_dict)}{
token\_pairs = [pair for pair \textbf{in} token\_price\_dict \textbf{if} pair[0] != ``00000000"]\;
    total\_volume = 0\;
\For{\textbf{each} pair\_info \textbf{in} token\_price\_dict.values() }{
total\_volume = total\_volume + pair\_info[`volume']\;
}
\For{\textbf{each} pair \textbf{in} token\_pairs}{
pair\_info = token\_price\_dict[pair]\;
weight = pair\_info[`volume'] / total\_volume\;
pair\_info[`weight'] = weight\;
}
\For{\textbf{each} pair\_info \textbf{in} token\_price\_dict.values()}{
pair\_info[`inv\_pct\_mov'] = -pair\_info[`pct\_mov']
}
bdr\_pct\_mov = 0\;
\For{\textbf{each} pair\_info \textbf{in} token\_price\_dict.values()}{
bdr\_pct\_mov = bdr\_pct\_mov + (pair\_info[`inv\_pct\_mov'] $*$ pair\_info[`weight'])\;
}
bdr = 1 + (bdr\_pct\_mov / 100)\;
\textbf{return} bdr\;
}
\end{algorithm}

\begin{algorithm}[h]
\scriptsize
\DontPrintSemicolon
\SetKwInput{KwKey}{Key} % Set the Key
\SetKwFunction{Fspend}{spend}
\SetKwProg{Func}{Function}{:}{{end}}
\caption{Tax Script}
\KwKey{signature, amount, current\_exchange\_rate, preimage\_of\_signature, tax\_percent}
\KwOut{updated stateful contract for the sender \& new stateful contract for the receiver}
DataLen = 1\;
utxo\_amount $\gets$ initial\_amount\;
pubKey $\gets$ \text{pubkey of the sender}\;
initial\_exchange\_rate $\gets$ initial exchange rate of the token\; 
region\_code $\gets$ region code of the person\;
tds $\gets$ TDS \;
\Func {\Fspend(sig, amount, current\_exchangerate, tax\_percent, receiver\_pubkey, preimage)}{
\If{checkSig(sig, pubKey) \AND Tx.checkPreimage(preimage) \AND check\_regiontax(region\_code,tax\_percent)}{
scriptCode $\gets$ SigHash.scriptCode(preimage)\;
codeend $\gets$ \text{position where the opcode ends}\;
codepart $\gets$ scriptCode[:codeend]\;
percentage\_movement $\gets$ get\_percentage\_movement(initial\_exchangerate, current\_exchangerate)\;
\If {percentage\_movement $>$ 0}{
gains $\gets$ (percentage\_movement $*$ (tax\_percent $*$ 10$^{-2}$) $*$ utxo\_amount) /(percentage\_movement + 1)\;
spendable\_amount $\gets$ utxo\_amount - gains -tds\;
}
\Else{
spendable\_amount $\gets$ utxo\_amount - tds\;
}
\If {amount $\leq$ spendable\_amount \AND sender == pubKey \AND amount $\geq$ 0}{
utxo\_amount $\gets$ utxo\_amount - amount\; 
} 
}
updated\_script $\gets$ codepart + utxo\_amount+sender + current\_exchange\_rate + tds\;				 
new\_script $\gets$ codepart+utxo\_amount + receiver\_pubkey + current\_exchange\_rate + tds\;
hash $\gets$ sha256(updated\_script+new\_script)\;    
\If{hash == SigHash.hashOutputs(preimage)}{  
\textbf{true}\; 
}
}
\end{algorithm}


\end{document}

