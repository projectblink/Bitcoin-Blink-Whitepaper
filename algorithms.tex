\documentclass[a4paper,10pt,twocolumn]{extarticle}
\usepackage[a4paper , top=8mm, right=8mm,left=8mm, bottom=8mm]{geometry}
\usepackage{nopageno}
\thispagestyle{empty}
\usepackage{geometry}
\usepackage{amsmath} % for math and equations
\usepackage{amssymb} % for symbols
\usepackage[boxruled,lined]{algorithm2e} %algorithm
\usepackage{algorithmicx} %algorithm
\usepackage{algpseudocode}
%declarations

\algnewcommand{\algorithmicand}{\textbf{ and }}
\algnewcommand{\algorithmicor}{\textbf{ or }}
\algnewcommand{\OR}{\algorithmicor}
\algnewcommand{\AND}{\algorithmicand}
\algnewcommand{\var}{\texttt}
\usepackage{comment}
\usepackage{color}
\usepackage{float}
\usepackage{hyperref}
\SetAlCapNameFnt{\scriptsize}
\SetAlCapFnt{\scriptsize}

%Algorithm2e Declarations
\SetKwProg{Func}{Function}{:}{{end}}
\SetKwInput{In}{in}
\SetKwInput{Not}{not}
\SetKwInput{Return}{return}
\SetKwInput{Declare}{declare}
\SetKwInput{Set}{set}
\SetKwInput{KwInput}{Input} % Set the Input
\SetKwInput{KwOutput}{Output} % set the Output
\SetKwInput{KwPre}{pre} % Set the Pre
\SetKwInput{KwPost}{post} % set the Post

\definecolor{pagecolor}{rgb}{1,0.98,0.9}
\title{\vspace{-10mm} \footnotesize \textbf{REFERENCE ALGORITHMS} \vspace{3mm}\\
\scriptsize \vspace{-2mm} Purva Choudhari\thanks{https://github.com/Purva-Chaudhari}, Ajay Joshua\thanks{https://github.com/I-Corinthian} \vspace{1mm}\\
\url{https://github.com/projectblink}\vspace{-14mm}}
\date{\scriptsize \today \hspace{0.5mm} (Genesis Version) \vspace{-7mm}}

\algnewcommand\Signal{\textbf{signal}}
\begin{document}

\maketitle
\scriptsize
\begin{algorithm}[H]
\scriptsize
\DontPrintSemicolon
\caption{Path Finding}
\SetKwFunction{Findpath}{findPath}
\Func{\Findpath(fromNode, toNode)}{
\tcc{Find a route from from Node to to Node}
paths = getAllPaths(fromNode)\;
routes = [ ]\;
\For{path \textbf{in} paths}{
\tcc{Check if the path is connected to the destination node}
\If {path.toNode == toNode}{
\textbf{return} [path]\;
\tcc{Try to find a route from the destination node through this channel}
route = findPath(path.toNode, toNode)\;
\If{route is \textbf{not} None}{

\tcc{Add this path to the route}
routes.append([path] + route)\;}}
\tcc{Return the route}
\If{len(routes) $>$ 0}{\textbf{return} (routes)\;\Else{\textbf{return} None\;}}}}
\end{algorithm}

\begin{algorithm}[H]
\scriptsize
\DontPrintSemicolon
\caption{Onion Peeling}
\SetKwFunction{Onionpath}{onion\_path}
\Func{\Onionpath(mint\_hash, route)}{
\tcc{Get the next hop path in the route}

next\_path = route.pop()\;

packet = create\_onion\_packet(mint\_hash, next\_path)\;

\For{path \textbf{in} reversed(route)}{
eph\_key = generate\_ephemeral\_key()\;

packet = add\_path\_to\_onion\_route(path, eph\_key, packet)\;
}
send\_packet\_to\_next\_hop\_path(packet, next\_path)\;

response = receive\_response\_from\_next\_hop\_path()\;

\For{path \textbf{in} reversed(route)}{

response = decrypt\_response\_with\_ephemeral\_key(response, path, eph\_key)\;

}
\textbf{return} response\;
}
\vspace{2mm}
\tcc{Notes : The onion peeling algorithm is used to protect the privacy of the mint route, by encrypting the mint information multiple times, with each layer containing information for the next hop. As the payment packet is passed from hop to hop, each node removes a layer of encryption to reveal the next hop in the route.
\begin{itemize}
\item mint\_hash is the unique identifier for the minted transaction
\vspace{-0.2cm}
\item route is a list of the nodes in the mint route
\vspace{-0.2cm}
\item add\_path\_to\_onion\_packet function adds a new layer to the onion packet for the current hop
\vspace{-0.2cm}
\item ephemeral key will be used to decrypt the response from that hop \end{itemize}}
\end{algorithm}

\begin{algorithm}[H]
\scriptsize
\DontPrintSemicolon
\caption{Node Weights}
Algo\;
%Scan the blockchain for node join script with bandwidth x, weight 0
%Scan from genesis block to current block
% Add weight and store in temp in node itself
%If Sucessful block add weight by 0.0001*block size limit per sec
%If unsucessful block without fee snip add weight -0.02* block size limit per sec
%If Fork Proof provided add weight to the prover node +0.01 *block size limit per sec and add weight to the forker -0.01 * block size limit per sec 
%Loop it after every block is added
\end{algorithm}

\begin{algorithm}[H]
\scriptsize
\DontPrintSemicolon
Algo\;
% If block time is over or block size is capped or end snip received, add block to chain
% Update block by receiving last blocks updated hash proof
% If new proof of block n is heavy than old proof, update the block % Find the snips to remove by linearly hashing one by one snip and when the MCR output = new proof MCR output then reject remaining snips
% Update the block and add to chain
% Add the next new block now to the chain
% Calculate the ring size after a block is received and confirmed by next block
% Ring size > 1 < no of nodes in node join req
% If the confirmed block is forked then, reduce ring size by 1 int, end election
% If the confirmed block is not forked, add ring size by 1 int, which denotes two tails should be elected
% Set the ring size in int
% From all node joining scripts make a set A of bandwidths descending
% Remove current ring validators from it
% From all node joining scripts make a set B of with updated weights in descending
% remove current ring validators from it
%From previous ring validators of previous block calculate the mean value >= 0
% set the mean value as ring tail join requirement
% Current head shall calculate the h by MCR of the block + Linear Hash of Hash Proofs
% Set h as K
% Update set A by > join req
% Update set B by choosing set A pubkeys only
% Order set B by descending
% Find 2 pubkeys of MD160 hash in set B which is lesser than K
% If no Find 2 pubkeys of MD160 hash in set B which is greater than K
% When the pubkey is found
% Find their bandwidth from their node joining script
% Take the median value from it and keep it as block size limit per sec
% Now we have to find per block time to find per block size
%Take ring size number of blocks
%For new upcoming n block find the block time
% using ring size take all the previous block
% Find the median of it = block time
% From the block time, find the per block size
% give block size in bits / sec and block time in global hashes
% from block time global hashes calculate the individual block time count using its IHR
% Set a cap that not more than the individual block time the producer should produce
\caption{Snip Construction}
\end{algorithm}

\begin{algorithm}[H]
\scriptsize
\DontPrintSemicolon
\caption{Hash Proofing}
Algo\;
%Hashclock (single thread hashing) start with first snip arrival
%Rebuild the ring validators for the block
%From ring validators rebuild the block size and individual time in hash counts
%Snip + Preimage + Signature attach for the hash that produces from concatenating
%Send to next instructed node on the routing instruction
% If ind block time finish stop accepting snips and add that as block
% If block size is exceeding reject snips and add that as block
\end{algorithm}

\begin{algorithm}[H]
\scriptsize
\DontPrintSemicolon
\caption{Hash Reward}
% Calculate Total hashes H and Total bitcoins g remaining for the halving period
% Per hash reward from total hash count (total sec * legacy hashrate) till next halve will have g bitcoins, if fork slash, if not rest counts will be added to g, Per hash reward = g/total remaining count
\end{algorithm}

\begin{algorithm}[H]
\scriptsize
\DontPrintSemicolon
\SetKwInput{KwSigIn}{signal input} % set the Signal Input
\SetKwInput{KwSigOut}{signal output} % set the Signal Input
\tcc{Public signals}
\KwSigIn{node\_ihr}
\KwSigIn{ihr\_hash}
\tcc{Private signals}
\KwSigIn{salt}
\KwSigIn{required\_ihr}
\tcc{Output signal}
\KwSigOut{if\_pass}
\tcc{Range proof check}
\textbf{signal}  buffer\;
\textbf{signal}  range\_check\;
\If{node\_ihr $>$ required\_ihr - buffer \AND node\_ihr $<$ required\_ihr + buffer} {range\_check = true\;}
\tcc{Verify hash}
\textbf{signal}  hash\;
\textbf{signal}  hash\_check\;
\tcc{RIPEMD160 to calculate the hash}
hash = RIPEMD160 (salt, required\_ihr)\;
\If{hash == ihr\_hash} {hash\_check = true\;}
\uIf{range\_check \AND hash\_check} {if\_pass = true\;}
\Else{if\_pass = false\;}
\tcc{Bandwidth circuit $\equiv$ IHR circuit}
\caption{ZK IHR Circuit}
\end{algorithm}

\begin{algorithm}
\scriptsize
\DontPrintSemicolon
class MerkleChain\;
\KwPre{the snip is added to the data}
\KwPost{the data is added to the chain}

 add\_node(snip)\;
 d $\gets$ snip\;
  \uIf{head = null}{
    head,tail $\gets$ add\_data(d)\;}
    \Else {tail $\gets$ add\_data(d)\;}

\hrulefill

class add\_data(d)\;
\KwPre{the value is added to the vector}
\KwPost{the vector is generated to a merkle tree and added to the chain}

New Vector data\;
data $\gets$ d\;
\If{size(data) == max\_block\_size}
{generate\_root(data)}

\hrulefill

generate\_root()\;
\KwPre{the vector data is added as the leaves}
\KwPost{merkel tree and its root is generated }

New Vector temp\_data\;
temp\_data $\gets$ data\;
\While {temp\_data $>$ 1}{
\For{i = 0  i $<$ size(temp\_data)  i+2}{
Left $\gets$ temp\_data[i]\;
Right $\gets$ (i+1 == size(temp\_data)) ? temp\_data[i] : temp\_data[i+1]\;
combined = Left + Right \;
new\_temp\_data $\gets$ hash(combined)\;
}
temp\_data $\gets$ new\_temp\_data\;
}
node\_root $\gets$ temp\_data[0]\;

\hrulefill

main()\;
\SetKwInput{KwInitialized}{initialized} % set the Initialized
\KwInitialized{chain is an object of class MerkleChain and  string data}

\While {true}
{Output “enter data (q to quit)”
Get data\;
\If {data = q}
{ 
\textbf{break}\;
\Else {
addnode(data)\;}
      }
     }
\caption{Merkle Chain}
\end{algorithm}

\begin{algorithm}[H]
\scriptsize
\DontPrintSemicolon
\Linesnumbered
\caption{Open Order Swap Script}
\SetKwFunction{Order}{order}
\SetKwFunction{Claim}{claim}

\textbf{declare} token\_a \textbf{as} integer\;

\textbf{declare} seller \textbf{as} PubKey\;

\textbf{declare} token\_b \textbf{as} integer\;

\textbf{declare} mature\_time \textbf{as} integer\;

\textbf{set} mature\_time \textbf{as} expiry\_time\;

\Func{\Order(sig, b, buyer, current\_exchange\_rate\_value, preimage)}{
\If{mature\_time $>$ SigHash.nLocktime(preimage)}{
\If{checkSig(sig, buyer)}{
\If{Tx.checkPreimage(preimage)}{
\If{b == this.token\_b}{
scriptCode = SigHash.scriptCode(preimage)\;

codeend = 104\;

codepart = scriptCode[:104]\;

outputScript\_send = codepart + buyer + num2bin(this.token\_a, 8) + num2bin(current\_exchange\_rate\_value, 8) + num2bin(tds, 8)\;

output\_send = Utils.writeVarint(outputScript\_send)\;

outputScript\_receive = codepart + this.seller + num2bin(this.token\_b, 8) + num2bin(current\_exchange\_rate\_value, 8) + num2bin(tds, 8)\;

output\_receive = Utils.writeVarint(outputScript\_send)\;

hashoutput = hash256(output\_send+output\_receive)\;
\If{hashoutput == SigHash.hashOutputs(preimage)}{
\tcc{order is open & placed}
}
}
}
}
}
}
\hrulefill

\Func{\Claim(sig, value, pubKey, current\_exchange\_rate\_value, preimage)}{
\If{mature\_time $<$ SigHash.nLocktime(preimage)}{
\If{pubKey == this.seller}{
\If{checkSig(sig, pubKey)}{
\If{Tx.checkPreimage(preimage)}{
\If{value == this.token\_a}{
scriptCode = SigHash.scriptCode(preimage)\;

codeend = 104\;

codepart = scriptCode[:104]\;

outputScript\_claim = codepart + pubKey + num2bin(this.token\_a,8) + num2bin(current\_exchange\_rate\_value,8) + num2bin(tds, 8)\;

output\_claim = Utils.writeVarint(outputScript\_claim)\;

hashoutput = hash256(output\_claim)\;

\If{hashoutput == SigHash.hashOutputs(preimage)}{
\tcc{claim is successful}
}
}
}
}
}
}
}
\end{algorithm}

\begin{algorithm}[H]
\scriptsize
\DontPrintSemicolon
\caption{Exchange Rate Calculation}
% bitcoin 0 , other tokenid x
% Calculate Bitcoin Exchange rate from list of orders of all tokens
% For Bitcoin calculate by taking weighted average of all token pairs
% Set token id = bitcoin exchange rate
% Update for every x blocks
\end{algorithm}

\begin{algorithm}[H]
\scriptsize
\DontPrintSemicolon
\SetKwInput{KwKey}{Key} % Set the Key
\SetKwFunction{Fspend}{spend}
\SetKwProg{Func}{Function}{:}{{end}}
\caption{Tax Script}
\KwKey{signature, amount, current\_exchange\_rate, preimage\_of\_signature, tax\_percent}
\KwOut{updated stateful contract for the sender \& new stateful contract for the receiver}

	DataLen = 1\;
	
	utxo\_amount $\gets$ initial\_amount\;
	
	pubKey $\gets$ \text{pubkey of the sender}\;
	
	exchange\_rate $\gets$ initial\_exchange\_rate\;
	
	tds $\gets$ TDS \;

	\Func {\Fspend(sig, amount, current\_exchange\_rate, tax\_percent, receiver\_pubkey,preimage)}
		{

		\If{checkSig(sig, pubKey) \AND Tx.checkPreimage(preimage)}
			{

			scriptCode $\gets$ SigHash.scriptCode(preimage)\;

			codeend $\gets$ \text{position where the opcode ends}\;

			codepart $\gets$ scriptCode[:codeend]\;

			gains $\gets$ (amount * current\_exchange\_rate) - (amount * exchange\_rate)\;

			\If {gains $>$ 0}
				{

					amount $\gets$ amount - (gains*(tax\_percent/100))*(current\_exchange\_rate)\;

					\If {amount \leq (amount - tds) \AND sender == pubKey \AND amount \geq 0}
						{

						utxo\_amount $\gets$ utxo\_amount - amount\; 
						} 
				}
					
			updated\_script $\gets$ codepart + utxo\_amount+sender + current\_exchange\_rate + tds\;				 

			new\_script $\gets$ codepart+utxo\_amount + receiver\_pubkey + current\_exchange\_rate + tds\;
   
			hash $\gets$ sha256(updated\_script+new\_script)\;    
  
			\If{hash == SigHash.hashOutputs(preimage)}
				{  
				\textbf{true}\; 
}
}
}
\end{algorithm}


\begin{algorithm}[H]
\scriptsize
\DontPrintSemicolon
\caption{Staking Script}
Algo\;
\end{algorithm}

\begin{algorithm}[H]
\scriptsize
\DontPrintSemicolon
\caption{DAO Contracts}
% Adding contracts
% Contracts withdrawal by proving
% Paying dividend according to weightage
% Removing contracts and temp membership
\end{algorithm}

\begin{algorithm}[H]
\scriptsize
\DontPrintSemicolon
\caption{Token Minting Procedure}
\end{algorithm}

\begin{algorithm}[H]

\tcc{Transfer and renting fees can only be deployed after the stable coin algorithm is written.}
\scriptsize
\DontPrintSemicolon
\caption{Transfer and renting fees}
\end{algorithm}

\end{document}

