\documentclass[../Bitcoin Blink.tex]{subfiles}
\graphicspath{{\subfix{../assets/images/}}}
\begin{document}
\normalsize
\subsection{VDF Proofs}

A common hardware single threaded hash-rate in seconds () is taken where the IHR proofs synchronize all the nodes in multiples of () to identify time in means of () hash-rate per second. All nodes IHR proof provided along with its PubKey for identification.

VDF Proofs provide timestamps of external data by concatenating it to a continuous hashing function restricted in a single threaded operation. During Multi-core validation, it can prove that a delay in real-time happened and the exact delay value can be calculated in hashes. Since our nodes synchronize as a constant hash-rate network, we can convert eitherwise from hash-rates to seconds.

Each node will have a hash-clock that produces continuous hashes and oncatenates the external data received. Clocks are initiated with the arrival of first external data (transactions)

Transactions are divided as snips and a bunch of snips form a block. Each snip shall be the external data concated to the clock's hashes.

Proofs are attached along with the external data when sent to the next node. Proofs include pre images and its signature which includes the public key. A Block time in hash-count will be provided for each block. This hash-count is given in () where the nodes shall take it in multiples and integer. 

From snip 1 to last snip it should be inside the hash-count or else reject the upcoming snips. If the block size is more also reject the upcoming snips. This creates a intra-block fork and resolved in the next block. 

\end{document}
