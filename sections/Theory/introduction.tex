\documentclass[../Bitcoin Blink.tex]{subfiles}
\graphicspath{{\subfix{../assets/images/}}}
\begin{document}

\section{Introduction}
The Bitcoin Network \cite{nakamoto2008bitcoin} and other altcoin blockchains with newer consensus and programmable money are unable to compete with centralized payment providers in speed and volume due to their sheer inability to scale with centralization issues. Many consensus models rely on external validation concepts such as finding a nonce and proving stake instead of rules tied to propagation itself. Imposing heavy reliance on users to acquire native chain tokens diminishes the adoption of blockchain, thus keeping them far from the wonders of this technology. Decentralized networks can effectively adapt to users’ needs by 1. Increasing block size 2. Decreasing block time 3. Eliminating low-efficient nodes 4. Increasing the requirement for node joining 5. Choosing producers by performance. Retail Staking with non-custodial solutions encourages users to stake their bitcoin to become a world reserve currency for every financial instrument with an additional restrictive monetary policy that helps to reduce volatility in times of recession.

Instead of storing UTXOs for an indefinite period, which compromises storage, renting UTXOs and replacing them with a fingerprint after they have expired without altering the block’s Merkle root provides cheaper fees. With Bitcoin’s unlocking script and use of sCrypt \cite{sCrypt}, developers can create custom scripts with regulatory options involving various types of taxes within its UTXOs, performing identity verification off-chain, with signatures instructing nodes to validate regulated payments with self-custody of tokens. Decentralized Altcoins can be bridged one-way to facilitate payments. A floating rate stable coin \cite{stablecoin} without pegging to fiat can be issued with bitcoin as collateral for staking and yielding fees in future. Basic banking solutions can be developed in Bitcoin Script, whereas common computable programs can be deployed to a Layer 2 State Machine \cite{wood2014ethereum}, where nodes update the state by providing a Proof-of-Fee-Receipt paid in any token in the Bitcoin network.

\end{document}