\documentclass[../Bitcoin Blink.tex]{subfiles}
\graphicspath{{\subfix{../assets/images/}}}
\begin{document}
\subsection{Rewards}
Rewards are given for each snip hash concated with transactions validated by VDF proofs. Since newly mined bitcoins (5BTC/10mins) for a period of time are definite and each block's time is capped in blink implementation, new bitcoins can be supplied exactly proportional to the current Bitcoin issuance rate with halving. Halving of Bitcoin issuance is done every $R \cdot (1.26 \cdot 10^8)$ hashes, where R is the common hardware's single thread hashrate. Each VDF hash represents work done by nodes on validating and propagating transactions. When a fork arises due to rejected snips, the rest of the block time and its allocated rewards are slashed. Ring validators also propagate proof for empty hashes without snips within themselves for accurate measures. Meanwhile, when a block is fully minted, the rest of its allocated rewards are distributed to the current halving period. This incentivizes nodes to attest and receive snips at the earliest to get increased block reward in the next blocks. While Tax outputs are attached as zero input transactions (coinbase tx) within the snip it contains, the rewards and fee outputs created as a separate snip denotes the end of a block.

During staking, producers announce their accepted tokens for which they will directly withdraw the commission. For other tokens, delegators can stake with a condition that their stake in bitcoins will be traded for the collected fees. During the commission withdrawal of non-accepted tokens, the producer will deposit collected fees to delegators and inflate the stake 1:1 ratio to withdraw the collateral. Users can pay in any token, delegators incur the risk, and producers get paid in tokens of their choice to validate transactions.
\end{document}
