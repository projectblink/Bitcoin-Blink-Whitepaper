\documentclass[../Bitcoin Blink.tex]{subfiles}
\graphicspath{{\subfix{../assets/images/}}}
\begin{document}

\section{Bitcoin (Blink)} 
\subsection{Election}
Block size denotes the amount of data that can be propagated across every node on the Bitcoin network hence, its success rate is directly dependent on the bandwidth each node allocates for confirmed block transmission. Block size is not limited and is fixed in every new block, which denotes that every validator of the block can send and receive the data size. Variable block size helps to scale the network by increasing transactions per block when the nodes upgrade and announce their bandwidth. A ZK-SNARK-based proof takes a node’s bandwidth as a public input signal, which is compared to the threshold range of the desired bandwidth of the network. To prevent tampering, the node will commit a salt, which is a large random number that will be added to the bandwidth to generate the $xyz$ hash which will be verified in the proof. Thus, the proof will attest its bandwidth output to the UTXO’s script and get validated. 

%Diagram Here
The network in consensus can forbid low bandwidth nodes from participating in the election to produce blocks. Bandwidth requirements are increased for nodes to get elected for next block production can assure the capacity to hold more transactions to scale further. Elections will be conducted based on nodes bandwidth and each node's honesty weight. Nodes identities are masked by their public keys encouraging privacy. For each block, the elected node will produce, and a finite \& unique set of nodes will validate and attach to the longest chain \cite{nakamoto2008bitcoin} . This set of nodes of following blocks $n \text{ to } m$ are called as a "ring". After each block's confirmation, the head of the ring eliminates itself and elects a tail which can be validated by nodes during propagation. The ring head after receiving the confirmed block will acquire bandwidth to gossip it's block over to other full or SPV nodes. The election seed is based on Merkle Chain root and VDF Proof \cite{yakovenko2018solana} of the newest confirmed block which is constructed by validators after resolving forks is taken as a random input value. Thus, the election is conducted for every block where a tail node is assigned.

\end{document}
