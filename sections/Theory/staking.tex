\documentclass[../Bitcoin Blink.tex]{subfiles}
\graphicspath{{\subfix{../assets/images/}}}
\begin{document}

\subsection{Staking}
The chain native token - bitcoin can be staked for node's public keys with specified token IDs, where the collateral can be used only once for a block. This results in a stake per token per block bringing the throughput (tps), per token basis. Each token per block's collateral requirement ($y$) is given in market price (USD) by taking the mean volume of previous blocks. Staked bitcoins can be withdrawn at any time, with no vesting period, except when the block is being created. This brings in a retail and non-custodial solution as opposed to security deposit-type PoS chains. Delegators won't lose their stakes as slashing is done directly on the fees during forks. For specific nodes bitcoins can be staked to collateralize or lock in its allocated block. Clients can provide their general users with a savings wallet in which they can benefit from an annual percentage yield by providing liquidity for staking. In this way, for a specific token's transaction to be included in a block, the first transaction should prove the collateral. Additionally an interest rate ($r$) is added along with the stake to restrict the free flow of bitcoins to increase demand in exchanges with a timelock of 21 days. This interest rate is determined by analyzing the market price of bitcoin from previous blocks. In future, a new collateralized floating-rate coin can be issued from lending L1 altcoins which can be used for staking to receive yield benefits. 
\end{document}
