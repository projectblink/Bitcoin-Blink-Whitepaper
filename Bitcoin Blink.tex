\documentclass[a4paper,10pt]{article}
\usepackage[a4paper , top=36mm, right=35mm,left=35mm, bottom=36mm]{geometry}
\usepackage{hyperref}
\usepackage{xcolor}
\usepackage{subfiles} % subtex files package
\usepackage{graphicx} %figures and images
\graphicspath{{assets/images/}} % file path for images
\usepackage{blindtext} % for blind text
\usepackage{float}% float package
\usepackage{enumitem} % for enumerate & itemize
\usepackage{amsmath} % for math and equations
\usepackage{tabularx} % tables
\usepackage{adjustbox} %to adjust boxed content
\usepackage{tikz} % for tikz picture, graphs, etc
\usepackage{pgfplots} % normal/logarithmic plots in two and three dimensions
\usepgfplotslibrary{external} % pgfplots libraries
\usepackage[edges]{forest}  % for classification trees
\usetikzlibrary{shapes.geometric} % shapes tikz 
\usetikzlibrary{arrows.meta,arrows} % arrows for trees 
\usepackage{listings} %listings
\usepackage{xurl}
\usepackage{indentfirst}
\renewcommand{\abstractname}{\vspace{-\baselineskip}}



\hypersetup{
	colorlinks = true,
	linkcolor = blue,
	urlcolor = purple,
	citecolor = red,
    pdftitle={Bitcoin (Blink) - Whitepaper},
    pdfauthor={},
    pdfsubject={blockchain},
    pdfkeywords={blockchain, bitcoin, cryptocurrency,proofofspeed,projectblink,blink},
    bookmarksnumbered=true,
    bookmarksopen=true,
    bookmarksopenlevel=1,
    pdfstartview=Fit,
    pdfpagemode=UseOutlines,
    pdfpagelayout=TwoPageRight
}

\font\myfontt=cmr12 at 10pt

\title{
 \Large \textbf{Bitcoin (Blink) - A Scalable \& Adoptable \\
Peer to Peer Cash System}\\
\vspace{6mm}
\scriptsize [WORKING-DRAFT]
}
\author{ \myfontt Joby Reuben \\ \myfontt jobyreuben@gmail.com \and  \myfontt Purva Chaudhari \\ \myfontt purva@example.com}
\date{}
\begin{document}
\maketitle
\begin{abstract}
\noindent \textbf{Abstract :} Bitcoin's PoW is replaced with a propagation competition on blocks sent across validators under a certain time interval stamped with cryptographic proofs to claim fees and solve forks as per proof weight. To bring adaptable scalability the block sizes are decided on consensus among elected nodes of specific epochs to decrease waiting transactions. Gossip systems are replaced with a privacy-centered direct messaging system by constructing encrypted paths to deliver unconfirmed transactions \& confirmed blocks. Aside from bringing speed, we resolved the need for a single transaction fee token for a blockchain by bringing forth a novel non-custodial per-token staking system to offer users to pay in any token. Bitcoin as a currency will hold the security of the network, Layer-1 tokens with staking and yielding fees. Since Bitcoin script adapts a Turing-incomplete language and doesn't involve loops, the fees are imposed for renting UTXOs which makes transactions cheaper and the chain's ledger size optimized. We propose solutions for regulation revolving around taxation within the self-custody wallet ecosystem without compromising users' privacy. 

\end{abstract}

\section{Introduction}

Bitcoin Network and other altcoin blockchains with newer consensus and programmable money are unable to compete with centralized payment providers in volume due to their sheer nature of inability to scale with centralization issues. Rules imposing heavy reliance on users acquiring chain native tokens are adoption diminishing requirements which hide users the wonders of blockchain technology for different regions of the world. Decentralized networks can effectively adapt to users' needs by 1. Increase Block Size 2. Decrease Block Time 3. Eliminate Low Efficient Nodes 4. Increase Joining Requirement. Retail Staking with non-custodial solutions encourages users to stake their Bitcoin to become a world reserve currency for every financial instrument with an additional restrictive monetary policy that helps to reduce volatility at time of recession.

Instead of storing UTXOs for an indefinite time which compromises storage, renting UTXOs and replacing them with a fingerprint after it expires without altering the block's Merkle root, thus providing cheaper fees. With Bitcoin's unlocking script and use of sCrypt - a high level language, developers can create custom scripts with -regulatory options involving various types of taxes within its UTXOs, offloading identity verification off-chain, with signatures instructing nodes to validate regulated payments with self-custody of tokens. Basic Banking solutions can be developed in Bitcoin Script whereas common computable programs can be deployed to Layer 2 EVM State Machine which updates the state by providing a Proof-of-Fee-Receipt paid in Bitcoin Layer. 

\section{Election}
Block size denotes the size of data that can be propagated across every producer node on the Bitcoin network, hence it's success rate directly dependant on the Bandwidth each node allocates for confirmed blocks transmission. Block size is not capped, but fixed every n epoch which validates that every producer node on the network can send and receive the data size. Variable Block Size helps in scaling the network by increasing transactions per block if nodes upgrade and announce their bandwidth. Bandwidth's can be proved.....@Purva

A vote can be taken across producer nodes if there are increase in unconfirmed transactions cannot fit into a block. The network in consensus can forbid low bandwidth producer nodes participating the election, thus increasing the joining requirement and capacity to hold more transactions. Votes can be published onchain by.....@Purva

As Bandwidth plays major role in scalable infrastructure, nodes are required to have better of it to achieve maxium production rate per epoch, as elections will be conducted based on it and each node's honesty weight. Every Node willing to participate in the next epoch block production, identity is given in its public keys published to the ledger onchain for definite calculations.

To randomize the random seed which commences election, shall be identified from epoch's range of block's (n-m) Block Merkle Chain Root which is constructed by validators

Predictable Election Result, Producer Signature


\section{Staking}

Bitcoins can be staked for public keys with specified token id where the collateral can be used only once for a block. This results in stake per token per block. Each tokens per block collateral requirement is given in exchange rates by taking the median volume of all the blocks of the previous epoch.

Staked Bitcoins can be withdrawn anytime, without vesting period except at the time of producing block bringing retail and non-custodial solution as opposed to security deposit type PoS chains. Bitcoins can be staked to a specific node which choses to include the stake by collateralizing or locking in its allocated block.

 Additionally a new collateralized stable coin can be issued which can be used for staking where other decentralized altcoins can be utilized to receive yield benefits. 

\section{Regulation}
Regulating cryptocurrencies via centralized exchanges \& custodians risk funds and doesn't encourages self-custodial ecosystem. A regulator must have authority to sign/approve transactions. Whitelisting specific hashed addresses belonging to specific countries verified and signed by Government asssigned Client Wallets or Regulators by either doing full KYC or minimal such as Mobile Number based OTP verification could work with maximum privacy. 

UTXOs are stamped with region proof on its unlocking script based on specific spending conditions that will only allow a transaction onchain if a taxes are deducted properly. Bitcoin scripts can work efficiently and securely as opposed to Turing complete smart contracts in this case. Tax models such as Capital Gains Slabs can be issued by governments independantly trustlessly and is validated in script execution. External taxes such as TDS, Sales tax can be imposed offchain independantly as its flexible to do so.

\section{Oracles}

\section{Messaging}
Unconfirmed Tx, Network Graph, Psedonymous Identities, Path Finding, Encryption, Mempools

\section{Propagation}
Snips, Blinks, Legates, Blink time, Competition, Kamikaze Proof, Hash clocks, Ring Propagation, Node Weight

\section{Rewards}
How rest Bitcoins are rewarded per hash, 21 million cap. Rewards limited for each hash, no tx no reward. Coinbase rewards, Commission Trade deal

\section{Renting}
Merkle Chain, Fingerprint, Expiry Value, Rent Rates, Transfer Fees, Single UTXO, State Update

\section{Tokens}
Layer 1 Tokens, One-way Bridges, Proof of Burn, Creation of new tokens to transact one should stake for it.

\section{Banking}
Exchanges, Lending \& Borrowing, Insurance, Mirrored Wallets, Decentralized Stable Coin

\section{Privacy}
Obscuring Amounts, Ring Signatures decryption by government

\section{Cash}
Layer 2 Cash System 

\section{Computing}
EVM Layer with unique Proof of Fee receipt consensus model, removing EOAs, Balances, Purely for Logic and State update. NFTs, 

\section{Maintenance}
Active Development Funds, Validators commission, DAO setup, sustainability.

\bibliographystyle{plain} %bibiliography package
\bibliography{citation.bib} %bib file

\appendix

\section{One}

\section{Two}

\end{document}